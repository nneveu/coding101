\documentclass[xcolor=dvipsnames, professionalfonts,t]{beamer} 
\usetheme{Dresden} 
\usepackage{url}
\setbeamersize{text margin left=10pt}
\setbeamersize{text margin right=10pt}
%\usepackage{beamerthemesplit}              %
\beamertemplateballitem % fancy bullets and numbering

\setbeamertemplate{navigation symbols}{}   % suppress navigation symbols
\addtobeamertemplate{frametitle}{}{%
	\logo{Argonne_cmyk_black-eps-converted-to}

}

\title[September 2018]{\vspace{-1em}Coding 101 }
\author[K. Woodruff and N.Neveu]{{\Large{Katherine Woodruff} \\ \small{New Mexico State University} \\
		\url{kwoodruf@nmsu.edu} \\ \vspace{1em} \Large{Nicole Neveu} \\ \small{Illinois Institute of Technology}   \\ \url{nneveu@hawk.iit.edu}    }} \vspace{-5em}
\institute[] % (optional, but mostly needed)
{   
	  
}
% - Use the \inst command only if there are several affiliations.
% - Keep it simple, no one is interested in your street address.
\date{ September 30, 2018 \\
	\includegraphics[width=3cm,keepaspectratio]{/home/nicole/Documents/presentations/logos/Argonne_cmyk_black}%
	\hfill \hfill \hfill%
	\includegraphics[width=4cm,keepaspectratio]{FNAL}%
}

\begin{document}
%%%%%%%%%%%%%%
%%%%%%%%%%%%%%

\begin{frame}
\titlepage
\end{frame}
%%%%%%%%%%%%%%%%%%%%%%%%%%%%%%%%%%%%%%%%%%%%%%%%%%%%%%%%%%
\begin{frame}{Installing Python}
  Required software:
  \begin{itemize}
    \item Python 3
    \item Python packages:
    \begin{itemize}
      \item Matplotlib (for plotting)
      \item NumPy (math and science)
      \item H5py (data file handling)
    \end{itemize}
  \end{itemize}
  We recommend using Anaconda Python installer
  \begin{itemize}
    \item Installation documentation: \\
    \url{https://conda.io/docs/user-guide/install/index.html}
    \item Go to Windows, MacOS, or Linux link
  \end{itemize}
\end{frame}
%%%%%%%%%%%%%%%%%%%%%%%%%%%%%%%%%%%%%%%%%%%%%%%%%%%%%%%%%%
\begin{frame}
\frametitle{Exercise:}
\begin{itemize}
	\item Download or clone the following repository: 
	\begin{itemize}
		\item \url{https://github.com/nneveu/coding101}
	\end{itemize}
	\item Open the "\textbf{emittance\_calc.py}" file
	\item Begin exercise!
	\item N = number of particles
	\vspace{-1em}
	
	\begin{equation}
		\text{\textcolor{blue}{Beam Size}} \quad \sigma_x^2 = \frac{\Sigma x^2}{N} - \left(\frac{\Sigma x}{N}\right)^2
	\end{equation}
	\begin{equation}
	\text{\textcolor{blue}{Momentum}} \quad \sigma_{p_x}^2 = \frac{\Sigma p_x^2}{N} - \left(\frac{\Sigma p_x}{N}\right)^2
	\end{equation}
	\begin{equation}
	\text{\textcolor{blue}{Corelation}} \quad \sigma_{xp_x} = \frac{\Sigma \left(x\,p_x\right)}{N} - \left(\frac{\Sigma x}{N}\right)\left(\frac{\Sigma p_x}{N}\right)
	\end{equation}
	\begin{equation}\label{eq:emittance}
	\begin{aligned}
	\text{\textcolor{blue}{Emittance}} \quad
	\epsilon_n = \sqrt{\sigma_x^2\, \sigma_{p_x}^2 -\sigma_{xp_x}^2}
	\end{aligned}
	\end{equation}
\end{itemize}
\end{frame}
%%%%%%%%%%%%%%


%%%%%%%%%%%%%%
\end{document}
